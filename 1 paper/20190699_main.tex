

\documentclass[12pt]{article}
%%%%%%%%%%%%%%%%%%%%%%%%%%%%%%%%%%%%%%%%%%%%%%%%%%%%%%%%%%%%%%%%%%%%%%%%%%%%%%%%%%%%%%%%%%%%%%%%%%%%%%%%%%%%%%%%%%%%%%%%%%%%%%%%%%%%%%%%%%%%%%%%%%%%%%%%%%%%%%%%%%%%%%%%%%%%%%%%%%%%%%%%%%%%%%%%%%%%%%%%%%%%%%%%%%%%%%%%%%%%%%%%%%%%%%%%%%%%%%%%%%%%%%%%%%%%
\usepackage[authoryear]{natbib}
\usepackage{pdflscape}
\usepackage{lscape, comment}
\usepackage{setspace}
\usepackage{fullpage}
\usepackage[reqno]{amsmath}
\usepackage{amssymb,enumerate}
\usepackage{dcolumn}
\usepackage{multirow}
\usepackage{url}
\usepackage{rotating}
\usepackage{float}
\usepackage{cite}
\usepackage{color}
\usepackage{breqn}
\usepackage{amsthm}
\usepackage{titlesec}
\usepackage{changepage}
\usepackage[colorlinks]{hyperref}
\usepackage[right=1in,left=1in,top=1in,bottom=1in]{geometry}
\usepackage[all=normal,title,sections]{savetrees}

\setcounter{MaxMatrixCols}{10}
%TCIDATA{OutputFilter=LATEX.DLL}
%TCIDATA{Version=5.50.0.2960}
%TCIDATA{Codepage=1252}
%TCIDATA{<META NAME="SaveForMode" CONTENT="1">}
%TCIDATA{BibliographyScheme=BibTeX}
%TCIDATA{LastRevised=Tuesday, January 29, 2019 14:44:31}
%TCIDATA{<META NAME="GraphicsSave" CONTENT="32">}

\doublespacing
\newcolumntype{.}{D{.}{.}{-1}}
\newcolumntype{d}[1]{D{.}{.}{#1}}
\newcommand{\R}{R}
\setlength{\topskip}{0pt}
\setlength{\headsep}{0pt}
\newtheorem{theorem}{Proposition}
\definecolor{dark-red}{rgb}{0.4,0.15,0.15}
\definecolor{dark-blue}{rgb}{0.15,0.15,0.45}
\definecolor{medium-blue}{rgb}{0,0,0.5}
\hypersetup{colorlinks, linkcolor={medium-blue},citecolor={medium-blue}, urlcolor={medium-blue}}
\setlength{\parindent}{20pt}

\begin{document}

\title{`Mobile’izing Agricultural Advice: Technology Adoption, Diffusion, and Sustainability \thanks{Corresponding Author: A. Nilesh Fernando, Department of  Economics, University of Notre Dame, Notre Dame, IN 46556 (nilesh.fernando@nd.edu)} \thanks{%
This paper was previously circulated as `The Value of Advice: Evidence from
the Adoption of Agricultural Practices'}}
\author{Shawn A. Cole and \ A. Nilesh Fernando \thanks{This study was registered in the AEA RCT registry
as AEARCTR-0000678. This research was funded by research grants from USAID
and AusAID; Cole also acknowledges support from the Division of Faculty
Research and Development at Harvard Business School. Fernando acknowledges
support from the HKS Sustainability Science program. In 2015, Cole
co-founded a non-profit organisation, Precision Agriculture for Development, seeking to further test, and bring to
scale, mobile phone-based agricultural advice services; this non-profit
funded a post-doctoral position for Fernando. We thank the Development
Support Centre (DSC) in Ahmedabad and especially Paresh Dave, Natubhai
Makwana and Sachin Oza for their assistance and cooperation. Additionally,
we thank Neil Patel and Awaaz.De for developing and hosting the AO system and the Centre
for Micro Finance (CMF), Chennai and Shahid Vaziralli for support in
administering the evaluation. We thank Niharika Singh, Niriksha Shetty,
Ishani Desai, Tanaya Devi, HK Seo, and Chungeun Yoon for excellent research assistance. We
are especially indebted to Tarun Pokiya for suggestions, his agricultural
expertise and management of the AO system. This paper has benefited from
comments from numerous colleagues and participants at a variety of seminars.}}
\maketitle

\thispagestyle{empty}

\onehalfspacing

\begin{center}
\textbf{Abstract}
\end{center}


{\footnotesize Mobile phones promise to bring the ICT revolution to
previously unconnected populations. A two-year study evaluates an
innovative voice-based ICT advisory service for smallholder cotton farmers
in India, demonstrating significant demand for, and trust in, new
information. Farmers substantially alter their sources of
information and consistently adopt inputs for cotton farming recommended by the service. Willingness to pay is, on average, less than the per-farmer cost of operating the service for our study, but likely exceeds the cost at scale. We do not find systematic evidence of gains in yields or profitability, suggesting the need for further research.}

{\footnotesize \noindent \textbf{JEL Classification Numbers:} O12, O13, Q16 }

{\footnotesize \noindent \textbf{Keywords:} Technology Adoption,
Agricultural Extension, Informational Inefficiencies }

{\footnotesize \doublespacing
\linespread{1.5} }

{\normalsize \cleardoublepage
\setcounter{page}{1} }

\section{\protect\normalsize Introduction}

{\normalsize 

Differences in technology adoption drive productivity differences in agriculture. In turn, a variety of observers have pointed out that access to information and
awareness of agricultural technologies may play an important role in
their adoption (\citealp{Jack_2011}). Yet in-person extension
services are expensive, slow, and cumbersome. In India, dispersed rural
populations, monitoring difficulties, and limited accountability severely
constrain the reach of in-person extension systems: fewer than 6\% of the
agricultural population reports having received information from these
services.\footnote{%
This estimate is from the 59th round of the National Sample Survey (NSS) and
asks farmers about their information sources for `modern agricultural
technologies'. See \citealp{GBA_2010} for a detailed discussion of this data.%
} }

{\normalsize The rapid spread of mobile phones holds the promise of bringing
high-quality advice to billions of previously unconnected individuals. Yet,
we know relatively little about whether individuals with low levels of
technological literacy will trust information and whether the provision of
information by phone will change behaviour. This paper examines whether the
introduction of an information service that is able to deliver timely,
relevant, and actionable information to farmers can meaningfully
influence agricultural practices. Specifically, we evaluate Avaaj Otalo
(AO), a mobile phone-based technology service that both pushes information to
farmers via voice calls, and allows users to call a hotline, ask questions,
and receive a recorded response from agricultural scientists and local
extension workers. Callers can also listen to answers to questions posed by
other farmers. }

{\normalsize Working with the Development Support Centre (DSC), an NGO with
extensive experience in delivering agricultural extension, the research team
randomly assigned toll-free access to AO to 400 households (hereafter, `AO
group') and to test the hypothesis that ICT\ would not be effective without
at least some in-person element, an additional 400 households received both
AO and an annual in-person extension session (hereafter, `AOE group'). A
further 400 households served as a pure control group. The households were spread across 40 villages in Surendranagar district in Gujarat, India, and randomisation occurred at the household level.}

{\normalsize The AO service also included weekly push content, delivering
time-sensitive information such as weather forecasts and pest planning
strategies directly to farmers. An important difference from prior ICT-based
agricultural extension programs is that the information was exclusively
delivered through voice messages as opposed to text-based approaches that
may be less suited to semi-literate environments. This paper presents the
results using three rounds of household surveys: a baseline, a midline one
year later, and an endline two years after the study began. To capture
information spillovers, all respondents were asked, prior to the
intervention, to identify the individuals with whom they discussed farming.}

{\normalsize We find considerable demand for the service: nearly 90\% of the combined treatment group (AO+AOE)
called into the AO line over two years and 40\% asked a
question. The average treatment respondent used the service
for almost 7 hours (median 5.2 hours), making 22 calls, calling into the
service for more than 2.5 hours and listening to 4 additional hours of push
content. In addition, we find that respondents in the AOE group used the
service for an hour more than the AO group (7.3 hours to 6.2 hours). The service
increased  subjective trust in mobile phone-based programs as a
source of agricultural advice, from nearly 0 points at
baseline to more than 6 points on a 10-point scale by the endline. }

{\normalsize Aggregating across survey rounds and treatments, we find a 5.8 standard deviation increase in
the reported use of mobile-phone based information in agricultural
decision-making, as measured by
an index aggregating a variety of input use and sowing decisions. Farmers
relied less on commissions-motivated agricultural input dealers for
pesticide advice, and less on their prior experience for fertiliser-related
decisions. Our outcomes are largely self-reported, raising the concern of 
desirability bias. Our survey teams were distinct from the service team (and not affiliated with the NGO). We are further reassured by the fact that farmers in 
treatment groups did not report any significant changes in the use of mobile
phone-based information for price information--which the service did not
provide; and that self-reported and server logged usage of the service are
virtually identical.} \footnote{We acknowledge, however, that this still leaves open the possibility of sophisticated demand effects where farmers over-emphasise the usefulness of features of AO.}

{\normalsize Relative to in-person extension programs (%
\citealp{Bardhan_2011}; \citealp{duflo2011nudging}) and interventions in
markets linked to agricultural productivity (\citealp{gine2009insurance}; %
\citealp{cole2013barriers}) that often involve costly on-site infrastructure
and labour resources, our intervention represents a potential low-cost way to promote technology adoption. However, even with considerable usage, the largely on-demand nature of the AO service means that respondents are not provided a homogeneous treatment, but rather recommendations tailored to their specific needs. This poses a challenge to evaluation, as it is difficult to specify ex-ante what one should expect the effect of the service to be, and because different farmers receive different messages. We address this by testing whether a broad set of agricultural practices, aggregated by crop-type and input-type, respond to the treatment.}

{\normalsize First, at the crop level, farmers who received access to the AO service were
significantly more likely to adopt recommended agricultural inputs for cotton cultivation -- their primary crop -- as measured by an aggregate index (0.13 sd). Looking at indices aggregating types of inputs, we find that the treatment increased the adoption of recommended seeds by 0.09 sd. In addition, accounting for spillovers these estimates increase to 0.15 sd and 0.11 sd for the cotton and seed indices, respectively.  Importantly, the
intervention induced consistently higher expenditure on irrigation (15\% relative to the baseline control mean) and seeds (20\% relative to baseline control mean), which we interpret as a complementary
investment to more input-intensive cultivation practices which were
recommended by the service. We find some evidence to suggest that the intervention influenced
 agricultural knowledge as measured by 44 questions (3\% overall and 5\% for cotton-related questions, relative to the baseline control mean). We suggest some caution in interpreting the estimates for inputs and knowledge: while the per-comparison p-values are significant (and relevant), adjusting these p-values for testing multiple hypotheses leaves them well above standard thresholds for rejection.}

{\normalsize The primary goal of our experiment was to evaluate how
technology could facilitate changes in technology adoption by farmers; in
this we find the program successful. We do not, however, find systematic evidence that the intervention increased crop yields or profit. While the point estimates are often positive, they are noisy, as smallholder yields, and particularly self-reported yields, are quite noisy \citep{lobell2018eyes}. In addition, we note that even where farmers follow practices that are beneficial in expectation,  the stochastic nature of rainfall complicates the detection of treatment effects \citep{rosenzweig2020external}. Ex-post, our single study is therefore likely underpowered to capture effects on these outcomes; we believe this is a crucial area for further study, including meta-analysis.}

{\normalsize Our treatment also induces variation in the availability of information 
in social networks, allowing us to estimate peer effects. Here, the evidence is mixed. We find some evidence for complementarities among treated respondents: 
treated respondents with more treated peers (social network members) are more likely to call into the service and use the service for longer. In addition, among a set of respondents who are peers of study respondents, we find that exposure to a treated respondent results in lower pest-related cotton losses.
On the other hand, we don't find that study respondents -- either in treatment or control -- exposed to the treatment through a peer, are more likely to change the sources of information they use in agricultural decision-making.}

{\normalsize At the time of the endline, we conducted a series of willingness to
pay (WTP) experiments to estimate demand for AO. Average willingness to pay for a
nine-month AO subscription across multiple price elicitation methods is
roughly \$2, compared to a cost of provision for the same period of \$7. While suggestive of low WTP, it is worth noting here that as this service scales, the per farmer cost of provision reduces substantially as the marginal cost of scaling is very low.\footnote{The non-profit organisation `Precision Agriculture for Development' (Cole is on the board of this organisation) estimates the current cost of approximately \$2 per farmer per year of a similar service serving almost 600,000 farmers in Odisha, India.}

{\normalsize A large literature focuses on the microeconomics
of technology adoption (for a survey, see \citealp{FR_2010}). We contribute to this literature by examining whether an information service can facilitate improved
production practices. Our work complements a developing literature on the potential for digital agricultural extension (see \citet{fabregas2019realizing} for a recent review). Prior experimental work has used SMS messages to send farmers agricultural advice: \citet{fabregas2018can} find that these messages increased the take-up of agricultural lime in Kenya and Rwanda, \citet{casaburi2014harnessing} find mixed evidence on sugarcane yields in Kenya, while \citet{fafchamps2012impact} do not find that SMS messages influenced cultivation practices in India. Among maize farmers in Uganda, \citet{van2017there} find that extension videos influenced cropping patterns. Our treatments differ from much of the previous work in that participants receive voice messages as opposed to SMS messages or videos. In addition, the flow of information is demand-driven and customised according to the needs of individual farmers rather than aggregated at the level of crop choice or region.}

More generally, this paper advances the literature on the
efficacy of agricultural extension (\citealp{Feder_1987}; %
\citealp{Gandhi_2009}; \citealp{DKR_2011}). The existing literature finds
mixed evidence of efficacy, though it is not clear whether this is due to
variation in programs offered or methodological challenges associated with
evaluating programs without plausibly exogenous variation (%
\citealp{Birkhaeuser_1991}). This paper complements evidence on the
historical efficacy of agricultural extension in promoting the adoption of
new agricultural technologies in India (\citealp{Bardhan_2011}) and provides
guidance as to lower-cost solutions for delivering advice. \citet{benyishay2013communicating} compare the impact of incentivised extension agents to non-incentivised extension
agents in Malawi, finding that incentives affect extension effort and that
the identity of the extension agent affects the adoption of information.}

{\normalsize We demonstrate that informational inefficiencies are real and
that farmers are aware they lack information:\ there is considerable demand
for high-quality agricultural information.\footnote{%
Informational inefficiencies in the context of technology adoption have been
defined as a situation in which farmers may not be aware of new agricultural
technologies or how they should be utilised (\citealp{Jack_2011}).} Our results complement recent work that measures
productivity enhancement from ICTs in developed countries (%
\citealp{draca2006productivity}).}

{\normalsize Finally, we carefully evaluate how selling the service, rather
than giving it away for free, would impact access to the service. We provide an experimental comparison of the Becker-DeGroot-Marschak (BDM) mechanism to standard
sales offers, demonstrating that the BDM\ mechanism, which requires a
smaller number of data points, yields credible estimates of the demand
curve.\footnote{In a companion paper, \citep{cole2015elicitation} we explore these findings in more detail.}}

{\normalsize This paper is organised as follows. The next section provides
context and the details of the AO intervention. Section 3 presents the experimental design and the
empirical strategy, while Section 4 presents the results from the two years
of survey data. Following this, Section 5 considers threats to the validity
of the results, and Section 6 concludes. }

\section{{\protect\normalsize Context and Intervention Description\label%
{Context}}}

\subsection{\protect\normalsize Agricultural Extension}

{\normalsize India is the second largest producer of cotton in the world,
after China.\ Yet, Indian cotton productivity ranks 78$^{th}$ in the world,
with yields only one-third as large as those in China. While credit
constraints, missing insurance markets, and poor infrastructure may account
for some of this disparity, a variety of observers have pointed out the
possibility that access to information and awareness of agricultural technologies may play an important role (%
\citealp{Jack_2011}).}

{\normalsize According to the World Bank, there are more than one million
agricultural extension workers in developing countries, and public agencies
have spent over \$10 billion dollars on public extension programs in the
past five decades (\citealp{Feder_2005}). The in-person extension model,
\textquotedblleft Training and Visit\textquotedblright\ extension, has been
promoted by the World Bank throughout the developing world and is generally
characterised by government-employed extension agents visiting farmers
individually or in groups to demonstrate agricultural best practices (%
\citealp{Anderson_2007}).\ Like many developing countries, India has a
system of local agricultural research universities and district-level
extension centres, producing a wealth of specific knowledge. In 2010, the
Government of India spent \$300 million on agricultural research and a
further \$60 million on public extension programs (RBI, 2010).}

{\normalsize For decades, the Government of India, like most governments in
the developing world, has operated a system of agricultural extension
intended to spread information on new agricultural practices and
technologies through a large work force of public extension agents.\
However, evidence of the efficacy of these extension services is limited. In
India, dispersed rural populations, monitoring difficulties, and a lack of
accountability hamper the efficacy of in-person extension systems: fewer
than 6\% of the agricultural population reports having received information
from these services.\footnote{%
This estimate is from the 59th round of the National Sample Survey (NSS) and
asks farmers about their information sources for `modern agricultural
technologies'. See \citealp{GBA_2010} for a detailed discussion of this data.%
} }

{\normalsize Yet, in-person extension faces several important challenges
that limit its efficacy:}

{\normalsize \noindent \emph{Spatial Dimension}: Limited transportation
infrastructure in rural areas and the high costs of delivering information
in person greatly limit the reach of extension programs.\ The problem is
particularly acute in interior villages in India, where farmers often live
in houses adjacent to their plots during the agricultural cycle, creating a
barrier to both the delivery and receipt of information. }

{\normalsize \emph{Temporal Dimension}: As agricultural extension is rarely
provided to farmers on a recurring basis, the inability of farmers to follow
up on information delivered may limit their willingness to adopt new
technologies. Infrequent and irregular meetings limit the ability to provide
timely information, such as how to adapt to inclement weather or unfamiliar
pest infestations. }

{\normalsize \emph{Institutional Capacity}: In the developing world,
government service providers often face institutional difficulties. The
reliance on extension agents to deliver in-person information is subject to
general monitoring problems in a principal-agent framework (\citealp{AF_2007}%
).\ For example, monthly performance quotas lead agents to target the
easiest-to-reach farmers and rarely exceed targets. Political capture may
also lead agents to focus outreach on groups affiliated with the local
government, rather than on marginalized groups for whom the incremental
benefit may be higher. Even when an extension agent reaches farmers, the
information delivered must be locally relevant and delivered in a manner
that is accessible to farmers with low levels of literacy. }

{\normalsize The importance of these constraints may be difficult to overstate (%
\citealp{Birkhaeuser_1991}; \citealp{Saito_1990}.) A recent nationally
representative survey shows that just 5.7\% of farmers report receiving
information about modern agricultural technologies from public extension
agents in India (\citealp{GBA_2010}.) This failure is only partly
attributable to the misaligned incentives of agricultural extension workers;
more fundamentally, it is attributable to the high cost of reaching farmers
in interior rural areas. }

{\normalsize Finally, a potential problem is that information provision to
farmers is often \textquotedblleft top-down.\textquotedblright\ This may
result in an inadequate diagnosis of the difficulties currently facing
farmers, as well as information that is often too technical for
semi-literate farming populations. This problem may affect adoption of new
technologies as well as optimal use of current technologies. }

{\normalsize In the absence of expert advice, farmers seek out agricultural
information through word of mouth, generic broadcast programming, or
agricultural input dealers, who may be poorly informed or face incentives to
recommend the wrong product or excessive dosage (\citealp{Anderson_2007}).%
\footnote{%
An audit study we conducted of 36 input dealerships in a block near our
study site provides a measure of the quality of advice provided by
commissions-motivated input dealers. Our findings suggest that the
information provided is rarely customised to the specific pest management
problems of the farmer and often takes the form of ineffective pesticides
that were traditionally useful but are no longer effective against the
dominant class of pests that afflict cotton cultivation. See \citep{cole2017promise} for further details. }}

{\normalsize These difficulties combine to limit the reliable flow of
information from agricultural research universities to farmers, and may
limit their awareness of and willingness to adopt new agricultural
technologies. Overcoming these \textquotedblleft informational
inefficiencies\textquotedblright\ may therefore dramatically improve
agricultural productivity and farmer welfare. The emergence of mobile phone
networks and the rapid growth of mobile phone ownership across South Asia
and Sub-Saharan Africa have opened up the possibility of using a completely
different model to deliver agricultural extension services. }

\subsection{\protect\normalsize Avaaj Otalo: Mobile Phone-Based Extension}

{\normalsize Roughly 50\% of the Indian labour force, or 250 million people,
are engaged in agriculture. As approximately 48\% own a mobile phone (as of
2015), mobile phone-based extension could serve as many as 120 million
farmers nationally\footnote{%
These figures are calculated using the Annual Report of the Telecom
Regulatory Authority of India (\citealp{india2015annual}) and the World Bank
Development Indicators (\citealp{world2014world}.). The WDI estimates the
rural population of India at 876 million, while the TRAI estimates the
number of subscriptions in rural India at 423 million. In addition, the WDI
estimate that 50\% of the workforce are engaged in agriculture out of a
total workforce of 497 million}. Mobile phone access has fundamentally
changed the way people communicate with each other and has increased
information flows across the country's diverse geographic areas. As coverage
continues to expand in rural areas, mobile phones carry enormous promise as
a means for delivering extension to the country's numerous small and
marginal farmers (\citealp{Aker_2011}).}

{\normalsize Our intervention utilises an innovative information technology
service, Avaaj Otalo (AO). AO uses an open-source platform to deliver
information by phone. Information can be delivered to and shared by farmers.
Farmers receive weekly push content, which includes detailed agricultural
information on weather and crop conditions that is delivered through an
automated voice message. }

{\normalsize Farmers can also call into a toll-free hotline that connects
them to the AO platform and ask questions on a variety of agricultural
topics of interest to them. Staff agronomists at the Development Support
Centre (DSC) -- our field partner -- with experience in local agricultural
practices receive these requests and deliver customised advice to these
farmers via recorded voice messages. Farmers may also listen and respond to
the questions their peers ask on the AO platform, which is moderated by DSC.
The AO interface features a touch-tone navigation system with local language
prompts, developed specifically for ease of use by semi-literate farmers.
The platform, which has now been deployed in a range of domains, was
initially developed as part of a Berkeley-Stanford research project on
human-computer interaction, in cooperation with the DSC in rural Gujarat (%
\citealp{Patel_2010}). }

{\normalsize Mobile phone-based extension allows us to tackle many problems
associated with in-person extension. AO has the capability to reach
millions of previously excluded farmers at a virtually negligible marginal
cost. Farmers in isolated villages can request and receive information from
AO at any point during the agricultural season, something they are typically
unable to do under in-person extension. Farmers receive calls with
potentially useful agricultural information on their mobile phones and need
not leave their fields to access the information. In case a farmer misses a
call, she can call back and listen to that information on the main line. AO
thus largely solves the spatial problems of extension delivery discussed
earlier. }

{\normalsize A considerable innovation of AO is tackling the temporal
problem of extension delivery. The agricultural cycle can be subject to
unanticipated shocks such as weather irregularities and pest attacks, both
of which require swift responses to minimise damage to a standing crop.
Because farmers can call in and ask questions as frequently as they want,
they can get updated and timely information on how to deal with these
unanticipated shocks. This functionality may increase the risk-bearing
capacity of farmers by empowering them with access to consistent and quality
advice. }

{\normalsize With respect to the problems of an institutional nature
mentioned earlier, AO facilitates precise and low-cost monitoring. The
computer platform allows easy audits of answers that staff agronomists
offer, greatly limiting the agency problem. Additionally, the AO system
allows for demand-driven extension, increasing the likelihood that the
information is relevant and useful to farmers. Push content is developed by
polling a random set of farmers each week to elicit a representative set of
concerns. In addition to this polling, the questions asked by calling into
AO also provide the information provider a sense of farmers' contemporaneous
concerns. This practice of demand-oriented information provision should
improve both the allocation and the likelihood of utilisation of the
information. }

{\normalsize However, while AO overcomes many of the challenges of
in-person extension, it eliminates in-person demonstrations, which may be
a particularly effective way of conveying information about agricultural
practices. As discussed in the following section, our study design allows us
to estimate the extent to which in-person extension serves as a complement
to AO-based extension, by providing a subset of farmers with both
in-person extension administered through staff at DSC and toll-free access
to AO.}


\section{{\protect\normalsize Experimental Design and Empirical Strategy\label%
{Design}}}

{\normalsize We selected two administrative blocks\footnote{%
A block is an administrative unit below the district level}, Chotila and
Sayla, in the Surendranagar district of Gujarat as the site of the study, as
our field partner, DSC, had done work in the area. Farmer lists, consisting
of all households that grew cotton and owned a mobile phone, were created in
40 villages and served as our sampling frame.}

{\normalsize We invited randomly selected farmers from this set to
participate in a study (farmers were not told that the study involved mobile
phones, nor were they told treatment status, when agreeing to participate in
the study). Nearly all agreed to participate, and we obtained a sample of
1,200 respondents, 30 from each village.\footnote{Online Appendix A5 provides further details on sample selection.} Fig. 1 summarises the experimental
design used in this study. Treatments were randomly assigned at the
household-level using a scratch-card lottery. The sample was split into
three equal groups. The first treatment group (hereafter, AOE) received
toll-free access to AO in addition to in-person extension. The in-person
extension component consisted of a single session each year lasting roughly
two-and-a-half hours on DSC premises in Surendranagar.\footnote{The in-person agricultural extension program was, in some sense, rather “light touch,” involving only a single meeting between the farmer and the extension team each year. This treatment arm was meant to address a potential concern about the AO service, that farmers would not trust a purely digital intervention.} The second treatment group (hereafter, AO) received toll-free access to AO, but no offer of
in-person agricultural extension, and the final set of households served
as the control group. In addition, among the two treatment groups (AO and
AOE), 500 were randomly selected to receive bi-weekly reminder calls
(hereafter, reminder group) to use the service, while the remaining 300 did
not. }

{\normalsize Fig. 2 provides a timeline for the study. Baseline data was
collected in June and July, 2011, and a phone survey consisting of 798
respondents was completed in November, 2011.\footnote{%
The previous version of this paper (\citealp{cole2012value}) analysed
treatment effects using results from this phone survey.} The midline survey
was completed by August, 2012, and the endline survey was completed by
August, 2013. }

{\normalsize To gauge balance, we compute a
simple difference specification of the form: }

{\normalsize 
\begin{equation}
y_{iv}=\alpha_{v}+\beta_{1}\ Treat_{iv}+ \varepsilon_{i}  \label{sd1}
\end{equation}
}

{\normalsize where $\alpha _{v}$ is a village fixed effect, $Treat_{iv}$ is
an indicator variable that takes on the value 1 for an individual, $i$, in
village $v$ assigned to a treatment group and 0 for an individual assigned
to the control group. We report robust standard errors below the coefficient
estimates. }

{\normalsize Because of random assignment, the causal effect of the
intervention can be gauged by comparing the treatment to the control mean. We use the ANCOVA specification as suggested by \citet{mckenzie2012beyond} in order to increase our power to detect effects, given the low autocorrelation of most outcomes in our data. Specifically, our main specification only uses the midline and endline data and controls for the baseline value of the outcome of interest:  

{\normalsize 
\begin{equation}
y_{ivt}=\alpha_{v} + \alpha_{t} +\beta_{1}\ Treat_{iv}+ \beta_{2}\ y_{iv0} + \varepsilon_{i}  \label{sd2}
\end{equation}
}

{\normalsize where $\alpha_{v}$ and $Treat_{iv}$ are as above, $\alpha_{t}$ is a fixed effect for the survey round and $ y_{iv0} $ is the baseline value of the outcome of interest.}

{\normalsize While increasing statistical power, the decision
to randomise at the household rather than village level raises the
possibility that the control group may also have access to information
through our treatment group. This suggests that any treatment effects may in
fact underestimate the value of the service.}


{\normalsize
In order to systematically assess this concern, we include a specification -- denoted as `Spillover' --  in all the main tables that controls for potential spillovers. Specifically, at baseline, we asked all respondents to list the three contacts with whom
they most frequently discussed agricultural information. As such, we are able to identify whether any of these peers also received the treatment. We amend the ANCOVA specification above to include a set of fixed effects for the number of peers listed at baseline and a control for the fraction of these peers who received the treatment as below: }  


{\normalsize 
\begin{equation}
y_{ivt}=\alpha_{v} + \alpha_{t} +\beta_{1}\ Treat_{iv}+ \beta_{2}\ y_{iv0} + \beta_{3} Treat\_Frac_{iv} + \sum(\#\ Peers = i)_{iv} + \varepsilon_{i}  \label{sd3}
\end{equation}}


{\normalsize where $\alpha_{v}$ and $\alpha_{t}$ are as above, $\sum I(\#\ Peers
= i)_{iv}$ is a fixed effect for the number of peers listed by a respondent at baseline and $Treat\_Frac_{iv}$ is the fraction of these peers who are assigned to
treatment.}

{\normalsize While we view the above specification as a `control' that allows us to compare how the treatment effect $\beta_{1}$ changes with and without spillover controls, we also view spillover effects as being of independent interest. } 

{\normalsize As such, in Table 6 we estimate the above specification for all study respondents and we separately estimate the effect of being exposed to a treated respondent for non-study respondents. 
In particular, using the `peer survey' we collected information 
on 1,114 non-study respondents, i.e. peers listed by study respondents at baseline who were not themselves a part of the study.\footnote{As we note in Figure 1, the peer survey included 1523 respondents, 409 of whom were study respondents and the remaining 1114 were non-study respondents. Of those who were study respondents, 143 belonged to the control group, 140 belonged to the AOE group, and 126 belonged to the AO group.} We estimate the extent of such peer effects or information spillovers with the following specification:}

{\normalsize 
\begin{align}  \label{sd4}
y_{iv} =\alpha_{v}+\beta (Treat\ References/References)_{iv} & + \sum_{i=2}^7 I(\#\ References = i)_{iv} + \varepsilon_{iv} & 
\end{align}
}

{\normalsize where $\alpha_{v}$ is as above, $\sum_{i=2}^7 I(\#\ References
= i)_{iv}$ is a fixed effect for the number of study respondents who list a peer 
as a top agricultural contact and $(Treat\ References/References)_{iv}$ is the fraction of these respondents who are assigned to treatment. }

{\normalsize We did not prepare a pre-analysis plan prior to undertaking the
study. This was in part due to the dynamic nature of the treatment:\ the
service responded to farmer questions and it was not always clear ex ante
which subjects farmers would inquire about. We address concerns about
multiple inference in four ways.\ First, we use the content generated by
farmers, and by our agronomist, as a broad guide for conducting empirical
analysis.\footnote{%
See online Appendix Table A1 for details of questions asked by farmers on the AO
service and push content provided.} Second, we aggregate agricultural practices into indices, following, for example, \citet{kling2007experimental}
Our agronomist characterised all agricultural practices reported in the survey as either consistent with best practices (we assign 1 to these responses), or either inconsistent or indeterminate (we assign a 0 to these responses). We then aggregate all variables corresponding to recommended practices by calculating a z-score for each component and take the average z-score across components. It is important to note that this is not a quantity index (i.e., more pesticide or fertiliser does not  increase the corresponding z-score in a deterministic manner). Rather, components of the z-score are positive for appropriate use of inputs. Each component z-score is computed relative to the control group mean and standard deviation at baseline. The components of the index are weighted by the inverse of the covariance matrix to adjust for highly correlated outcomes \citep{anderson2008multiple}. }

{\normalsize Third, we address the importance of Type 1 error directly in
two ways. First, in online Appendix A2 we report the number of comparisons that are
statistically different from zero at conventional levels of statistical
significance in each survey round and for each treatment arm. Panel A shows
that at baseline, the number of comparisons that are significantly different
from zero is consistent with what we would expect given Type 1 error at each
level of significance. In contrast, the analogous results for the midline
(Panel B) and endline (Panel C) reveal that we are able to the reject the
null for a much larger number of the same comparisons than would be
predicted by Type 1 error.}
 
{\normalsize Finally, we use both the standard Bonferroni-Holm and the Westfall-Young correction to compute a
family-wise error rate (FWER) across our main outcomes of interest (See
online Appendix A9). The latter correction uses randomisation inference to compute a family-wise error rate (FWER) for a set of comparisons. By re-estimating the full set of outcomes this correction takes into account the correlation between outcomes and is less conservative than the Bonferroni-Holm method. We separately compute a FWER for input adoption (Panel B) and for an overall set of summary indices (Panel A) \citep{anderson2008multiple}}.}

\subsection{\protect\normalsize Summary Statistics and Balance}

{\normalsize In this section, we assess balance between the treatment
group that received access to the advisory service and in-person extension
(\textquotedblleft AOE\textquotedblright ), the treatment group that only
received access to the advisory service (\textquotedblleft
AO\textquotedblright ), the combination of these two treatment groups (\textquotedblleft
Treat \textquotedblright) and the control group.}

{\normalsize Table \ref*{baseline} contains summary statistics for age,
education, profit from agriculture, and cultivation patterns for respondents in the study,
using data from a baseline paper survey conducted in July and August of
2011. Column (1) reports the mean and standard deviation for the control
group and column (2) tests the initial randomisation balance between the combined treatment group and the control group. Column (3) tests the initial randomisation balance between the AOE group and the control group, column (4) tests the balance between the
AO group and the control group and column (5) tests for balance between the AOE group and AO group. }

{\normalsize We see that respondents are on average 46 years old and have
approximately four years of education. Columns (3)-(5) show that the
randomisation was largely successful for the treatment groups across
demographic characteristics (Panel A) and indices capturing information
sources, crop-specific and general input use (Panel B). However, an
imbalance exists in the index for wheat management between the AOE group and control and another imbalance exists in the area of cotton planted between the AO group and the
control group.\footnote{Online Appendix A20 details the variables used to construct all aggregate indices.} However, the latter imbalance exists in 2010 but not in 2011 (both periods
are prior to treatment).\footnote{%
Note, the 2011 figures for wheat and cumin are not reported, as they are
grown during the Rabi season after the treatment was administered.} Both
treatment groups are also more likely to grow wheat, but this crop is mostly
grown for home consumption in this context.}

{\normalsize Particularly as cotton is the most important crop in our sample, we understand the importance of systematically accounting for baseline differences in covariates across treatment groups. As such, in the tables that follow we adopt the double LASSO machine learning approach (DML) to pick an optimal set of control variables as proposed by \citet{belloni2014inference}. Online Appendix A18 details the set of control variables (including their interactions) to which we apply the algorithm.\footnote{Additionally, 
online Appendix Table A2 provides a more systematic treatment of balance in our
sample. We look for significant differences in baseline characteristics
between the treatment groups and control respondents. Among the differences computed using the latter specification
(examining all 1,643 baseline variables), we find that 0.7\% are
significantly different from zero at the 1\% level, 4.3\% are different
at the 5\% level of significance and 8.6\% at the 10\% level. These
results suggest that the randomisation was successful and that the 
imbalances are a result of chance rather than any systematic mistake in the
randomisation mechanism.}}


\section{{\protect\normalsize Experimental Results\label{results}}}

{\normalsize In the results we detail below, we report estimates comparing the combined treatment group (`Treat') to the control group -- hereafter, the `treatment group' -- rather than its constituent treatment arms (the `AOE' group and the `AO' group).\footnote{We report the disaggregated results by both the AOE and AO treatments as well as the reminder treatment in online Appendix Table A12.} Similarly, we do not present the effects of the reminder treatment in the main tables. In both cases, this is to streamline the discussion and presentation of our results because the treatment effects for the AOE group and the AO group rarely differ, in addition to there rarely being a marginal effect of the reminder treatment over and above that of the combined treatment group. However, there are a few exceptions to this general characterisation which we discuss below.}

We acknowledge that this means we estimate a composite treatment effect that includes a weighted-average of the treatment groups.\footnote{As budgetary restrictions prevented a full 2x2 factorial design, we are unable to implement the corrective methods proposed by \citet{muralidharan2019factorial}. Moreover, we note that when we estimate the `long' model, the qualitative conclusions of our analysis rarely change.} We estimate treatment effects averaged across survey rounds (as in equation \eqref{sd2}) rather than show effects by round, aside from cases where the evolution of treatment effects (e.g. adoption of the service in Table 2) or the stochastic nature of results is especially salient (e.g. yield effects are dis-aggregated in online Appendix A17). 

In the tables that follow, unless otherwise noted, we report estimates from the ANCOVA specification from equation \eqref{sd2}, the spillover specification from \eqref{sd3}, and a simple difference specification \eqref{sd1} using the double LASSO procedure (hereafter, DML) to select control covariates.\footnote{Note, in the DML specification the algorithm chooses whether to include the baseline value of the outcome as a control.} 

\subsection{\protect\normalsize Take-Up and Usage of AO}

{\normalsize Table \ref*{ao_use} reports information on take-up and usage of the AO service at midline and endline.\footnote{Note, we disaggregate results by midline and endline here to show the evolution of usage.} While control respondents were not barred from AO usage, we did not inform them of the service, and those who did use it had to pay their own airtime costs. Only four control respondents called into the AO line by the midline and a further 25 had called in after two years. As a result, virtually all AO usage is accounted for by respondents in the treatment group. Note, as there is no baseline usage of the service the estimates here are simple differences rather than from the ANCOVA specification.}

{\normalsize Driven primarily by the \textquotedblleft
push\textquotedblright\ features, adoption of the service was broad and
deep. By the endline, nearly 90\% of treatment respondents had called into the service, and the mean total usage for the combined treatment group -- including both incoming calls and
time spent listening to push calls -- was 7 hours (median 5.2 hours). Note, in this case we do see a significant difference of roughly 1 hour between usage of respondents in the AOE group and the AO group (see online Appendix A12).}

{\normalsize Overall, we find that inbound usage was more concentrated:\  average time spent calling in by the endline was 2.7 hours with 22 calls, however the top 10\% of users accounted for 70\% of incoming
call time. In contrast, interest in information sent through push calls was
more evenly skewed:  treatment respondents listened to 55\% (median
58\%) of total push call content by endline.\footnote{This is calculated as follows: the total minutes of content listened to by all users, divided by the total minutes of push content that would have been heard had all users listened to the entirety of each message.}  By the endline, 40\% of treated respondents had asked a question about their agriculture on the system.\footnote{In online Appendix A12, we also find that the reminder treatment significantly
increased service usage: the reminder group had used the service almost an hour more on average by
midline and over 90 minutes more by endline, but were not statistically more
likely to call into the line. }}

{\normalsize Taken together, these results represent substantial induced usage
for treatment farmers. Additionally, these effects also mask important
temporal patterns shown in Fig. 3, which reports average service use by
month. We see that there was substantial usage across treatment arms during
the first six months after the intervention was administered. Following this
period, usage has been trending down, but with important spikes during
sowing times and harvest time. This figure is suggestive of users
acquiring a stock of knowledge and supplementing thereafter with dynamic
information needs throughout the season. }

{\normalsize In addition to providing a temporally relevant flow of
knowledge, 40\% of the combined treatment group received
customised answers to agricultural questions. Online Appendix Table A1 provides a
categorisation of the questions asked by treatment respondents during the
two years of service. (The categories are not mutually exclusive.)
Unsurprisingly, columns 3 and 4 show that most questions (50\%) relate to
cotton, a majority (54\%) focus on pest management, and these numbers are
relatively stable across both years. Table A1 also reports information on
the content of push calls (columns 5-8), which tended to provide more
information on cumin and wheat cultivation than incoming questions, and were
the primary source for weather information. A larger study might have
experimentally varied the topics of the push content (for example, either
matching or not matching subjects of contemporaneous queries), to further
disentangle the role of push vs. pull information. However, operational and
power considerations precluded inducing such variation in this study.}

\subsection{\protect\normalsize Sources of Information and Agricultural Knowledge}

{\normalsize Panel A of Table \ref*{info_gen} examines the use of mobile
phone-based information in agricultural decision-making, and measured trust
(on a scale of 1-10) of information provided by mobile phones. On average, 
treatment farmers are 66 percentage points (p.p.) more likely to
report using mobile phone-based information to make agricultural decisions.
The treatment effects on the reported level of trust in mobile phone-based
information are also substantially higher: approximately 5.8 points
greater on a 10-point scale. An index aggregating the importance of mobile
phone-based information (analysis of the topics comprising this index
follows immediately below) for all subject areas is 5.5 standard
deviations higher across the treatment group and is significant at the 1\% level. }

{\normalsize We asked farmers for their most important source of information
for a series of agricultural decisions. The survey responses are recorded as
free text, without prompting, and coded into categories by our data entry
teams. We present results across a variety of subject areas. Panel B of
Table \ref*{info_gen} shows that treated respondents 
consistently report using mobile phone-based information across a series of
agricultural decisions. In particular, we observe large effect sizes in
the case of pest management (17 p.p.) and smaller effects in the case of
fertiliser decisions (9 p.p.) and crop planning (5 p.p.). }

{\normalsize Other than input-related decisions, mobile phone information is
also used by the treatment group for other topics such as 
weather information (30\%). Importantly, we do not find any effect of our treatment
on the use of mobile phones for price information. The AO\ service never
provided price information. This helps address the concern that social
desirability bias may be contributing to our results. Accounting for spillover effects consistently, though marginally, increases coefficient estimates. Likewise, using  DML to pick control covariates leaves both the point estimates and the precision with which they are estimates largely unchanged.\footnote{Online Appendix A3 provides crop-specific results on sources of
information disaggregated by survey round. Treatment group respondents
report using information from input dealers less often in making cotton
pesticide decisions (-7.2\% at midline), although, interestingly, they
report consulting input dealers \textit{more} often in the case of cotton
fertiliser use (5\%) and cumin planting (3.7\%) at the endline. There are
also reported reductions in the use of information from \textquotedblleft
other farmers\textquotedblright\ and \textquotedblleft past
experience.\textquotedblright\ The reduction in reliance on past experience
for cumin fertilisers is significant at the midline.}}

{\normalsize Next, we ask whether this change in sources of information translates into changes in agricultural knowledge.  To do so, we examine whether AO improves farmers' ability to answer a set of 44 
basic agricultural questions. The questions test the respondents on a wide
range of topics, which are generally invariant to their personal
circumstances.\footnote{%
The full text of the questions is available in online Appendix A6.} }

{\normalsize Baseline agricultural knowledge is low, with farmers in the
control group only being able to answer 32\% of questions correctly. There
are no imbalances between treatment and control for the total at the
baseline. Given that these are very basic questions about agriculture, this
suggests that there is a substantial lack of information on even basic
topics concerning crop cultivation. }

{\normalsize In Panel C, we find that the main ANCOVA specification does not find a statistically significant difference in measured agricultural knowledge. However, in column 2, we see that accounting for spillovers yields a modest effect (3\% increase relative the control mean at baseline) and is significant at the 10\% level. Likewise, using DML increases the precision and yields a similar finding. For questions relating to the respondent's primary crop, cotton, we find that AO significantly increases knowledge by approximately 5\% and is statistically significant across all specifications.\footnote{Online Appendix A10 provides results on knowledge that are further disaggregated by question topic.}

{\normalsize Overall, these results suggest that the AO service was successful in establishing itself as a source of information for treatment respondents in making a variety of important agricultural decisions and produced modest gains in agricultural knowledge. In the next sections, we look at whether the provision of information through AO influenced input use and agricultural productivity. }

\subsection{\protect\normalsize Agricultural Input Adoption}


{\normalsize A number of input choices influence agricultural productivity.
Cotton is the main crop grown in our sample -- grown by 98.4\% of the
sample at baseline -- and chemical inputs such as pesticides and fertilisers
greatly affect cotton yields.\footnote{%
In 2006-2007, 87\% of all land under cotton in India was treated with
pesticide. In contrast, this figure is just 51\% for paddy and 12\% for
wheat. Calculations by author (Agricultural Census of India, 2006).} In
addition, Bt cotton is the dominant variety of cotton grown in this context
-- although there are literally hundreds of sub-varieties and brands which
pose other difficulties -- and yields are particularly sensitive to regular
irrigation.} Table \ref{input} tests whether the AO service influenced summary indices that capture recommended inputs for cotton, wheat, and cumin cultivation.  The recommended inputs include seed varieties, pesticides, and fertilisers (see online Appendix A20 for index components). 


\subsubsection{\protect\normalsize Input Adoption by Crop}

{\normalsize Panel A of Table \ref*{input} shows that the treatment increased the cotton management index by 0.12 standard deviations in the main ANCOVA specification (see online Appendix A20 for index components). This estimate rises to 0.14 standard deviations (significant at the 5\% level) when accounting for spillover effects, while the DML estimate is qualitatively similar.\footnote{Online Appendix A12 suggests the reminder treatment further increased compliance with cotton recommendations.}

In contrast, we do not detect significant differences for either the indices corresponding to  wheat management or cumin management across specifications.%
\footnote{%
The standard errors also suggest that the experiment may be underpowered to
detect effects for cumin (grown by just 34\% of the sample). Wheat
cultivation involves substantially fewer chemical inputs and is primarily
done for home consumption.} }


\subsubsection{\protect\normalsize Adoption by Input Type}

In Panel B of Table \ref{input} we test whether the treatment influenced input adoption as captured by summary indices that capture recommendations relating to seed, pesticide, and fertiliser management. We see that the index for seed management increases by 0.09 standard deviations (significant at the 10\% level), while controlling for spillover effects increases the size of this point estimate and its precision. However,  adding control covariates using DML reduces the point estimate.

In contrast, we do not find that the AO services influences the adoption of inputs relating to either pesticide or fertiliser management.

\subsubsection{Discussion of Input Adoption Results} 

The results above suggest that service was successful in changing the behaviour of farmers in cotton cultivation and, in particular, through influencing seed choice. In interpreting these results, we follow the framework used by \citep{kling2007experimental}. In our view, since the service was intended to influence cotton farming practices, the per-comparison p-values for the effects on the overall input adoption index are appropriate. However, in online Appendix A9, we adjust these asymptotic p-values for testing multiple hypotheses across a family of outcomes. In Panel B, we consider just the outcomes in Table \ref{input}. While p-value adjustments using the (conservative) Bonferroni-Holm method suggest there is a very high likelihood that the significant estimates could be observed by chance, adjusting for correlation between these outcomes using the Westfall-Young method substantially reduces the family-wise error. Nevertheless, even the Westfall-Young method suggests that one should interpret the input adoption results with some caution. 

While the indices above provided the most powered tests for behaviour change, here we provide some discussion on which inputs in particular drove these results.  The presence of a wide variety of cotton seeds, some counterfeit, makes seed selection a particularly important decision. In Uganda, \citet{bold2015low} demonstrate that low quality inputs depress returns to hybrid seeds.

{\normalsize In online Appendix A14 we report treatment effect estimates for components of the cotton index. We see a 6 p.p. increase (significant at the 1\% level) for Vikram, a Bt Cotton seed recommended by the service.  Similarly, looking at pest management we find that the service increase the use of a imidachlorpid by roughly 6 p.p. Imidachlorpid is a neonicotinoid pesticide that targets the nervous systems of insects and is a compliment to Bt cotton which has natural defences against bollworm. We also observe that the service increases the  adoption of Trichoderma, a biological method of pest control. The AO\ service provided extensive information in both Kharif and Rabi seasons on the use of Trichoderma as a means of preventing wilt disease in cotton and cumin. }

{\normalsize While we do not detect increases in fertiliser adoption across all crops, looking within the index we see that treated farmers increase their purchases of ammonium sulphate by approximately 3 p.p. and NPK Grade 1 by 4 p.p. across specifications. To put these effect sizes into perspective, \citet{DKR_2011} found an increase of
16-20\% in fertiliser adoption in Kenya using free delivery of planting and
top-dressing fertiliser, while \citet{benyishay2013communicating} found
increases of 2.2-5.5\% across treatments in pit planting and 0-19\% across
treatments for composting in a study using in-person extension. }

\subsection{\protect\normalsize Agricultural Productivity, Input Expenditure, and Profit}

{\normalsize In Table \ref*{sowing}, we examine
agricultural productivity, demand for the AO service, and profit. Overall, we do not find evidence to suggest that AO improved cotton, wheat or cumin yields. Importantly, as rainfall is highly stochastic and an important complement to chemical inputs like fertiliser, the treatment effects may vary by season \citep{rosenzweig2020external}. 

In online Appendix A17, we see that the point estimate on cotton yields, while insignificant, is positive at midline, but negative at the endline in the ANCOVA specification (columns 2-4). On the one hand, the low temporal correlation in yields suggests that an ANCOVA specification yields a more powerful test. However, baseline differences can more precisely be accounted for using a difference-in-difference specification. Indeed, doing so raises these point estimates and makes them consistently positive, though not statistically significant.\footnote{We note, however, that in online Appendix A12 we find some evidence to suggest that the reminder treatment influenced cotton yields, although at the same time, decreasing wheat yields.}  

With cumin, we see a statistically significant reduction in yield at midline and an increase at endline using the ANCOVA specification. Using a difference-in-difference specification, we find a similar statistically significant increase in yield at endline, though the reduction in yield at midline is now imprecisely estimated. Similarly, we find no evidence to suggest that the treatment influenced wheat yields across seasons. }

 In Panel B of Table \ref{sowing} we turn to how the treatment influenced input expenditure and profit. Total input expenditure includes outlays on seeds, pesticides, fertiliser, irrigation, hired labour and household labour priced at the mean wage of hired labour. We find that total input expenditure increased by nearly 8\% relative to the baseline control mean across specifications. Online Appendix A4 breaks down the treatment effects by input type and we find that this increase is driven by a roughly 15\% increase in expenditure on irrigation and expenditure on seeds which increased by roughly 20\% relative to the baseline control mean. Over a quarter of push calls contained information about weather forecasts, which provide farmers with important information on weather and reduce uncertainty about the timing and
value of irrigation. In addition, chemically intensive agricultural inputs
recommended by the service such as higher-yielding varieties of cotton are complements with increased irrigation.

We measure profits from agriculture by computing crop income and subtracting total input expenditure as defined above. While the point estimates for profit are consistently positive (ranging from 1.8\% when accounting for spillovers to 0.6\% in the DML specification, relative to profit for the control group at baseline) they are not statistically significant.  

Overall, these results suggest our experiment is likely underpowered to detect effects on yield and profit, which is further complicated by measurement error. We do note here, however, that this does not suggest that the effects are very large and a more highly powered study would detect large effects. Our baseline data suggests we can detect a 11.5\% increase in cotton yields and a 17.5\% increase in profits with 80\% power. However, as we discuss in Section 5.3, meta-analyses of such interventions suggest that the effects may be smaller; for smaller effects, our design has very limited power.

In the next section we discuss willingness to pay experiments for the service and what they reveal about how respondents value the information from AO. 

\subsection{\protect\normalsize Heterogeneous Treatment Effects}

{\normalsize In Online Appendix Table A11, we compare the combined treatment
group (i.e. AOE + AO) to the control group to investigate heterogeneity with respect to respondent education and income. As such, we modify equation \eqref{sd2} to include a dummy for being above the median of respondent education or income and the interact of the dummy with a treatment indicator. }

{\normalsize Treatment respondents with above-median incomes are no more
likely to call into the AO line, but their total usage is nearly one hour
greater than those with below-median income. Farmers with higher
incomes also show differential effects in the cotton management index (about
0.24 standard deviation units higher) and wheat management index (0.56 standard deviation units higher), both of which are statistically significant at the 10\% level).}

In contrast, we find little evidence to suggest the treatment has a differential impact on respondents by education. The one exception is that profit is significantly higher (7\% relative to the control mean) for those with below median education, but lower for those with above median education (3\% relative to the control mean). As we have elsewhere suggested that our experiment is likely not powered to pick up effects on profit, we do not view these results as persuasive but rather suggestive of the need for future work. 

\subsection{\protect\normalsize Spillover Effects}

{\normalsize Given randomisation at the household level, it is possible that
access to the service indirectly influenced the outcomes of study
respondents as well as those who were not a part of the study but in the
networks of study respondents through information spillovers.\footnote{In a separate paper, we document in detail how patterns of
social interactions and information exchange are influenced by the AO
treatment (\citealp{Fernando2016sotech}).}}

{\normalsize Table \ref{peer} estimates spillover effects for both study
respondents and a group of \textquotedblleft non-study\textquotedblright\
respondents who were surveyed in the \textquotedblleft peer
survey.\textquotedblright\ As the `Spillover' specification in tables 3, 4 and 5 suggest, the point estimates for estimated treatment effects often increase once we account for whether peers of study respondents were also exposed to the treatment. As such, they suggest treatment respondents may have discussed advice they received with their peers. Alternatively, peers may follow suit after directly observing changes in their neighbours' agricultural practices. }

In following section, we estimate spillover effects, using the specification in  
equation \eqref{sd4}. To wit, for study respondents we amend the main ANCOVA specification (equation \eqref{sd2}) to include a variable that captures the fraction of peers who they listed at baseline who also received the treatment (column 3) and a control variable for the number of peers they listed at baseline. Further, we include the interaction between their treatment status and the fraction of their peers who were treated (column 4). The latter helps us test whether there are complementarities among treated respondents.

For the non-study respondents we estimate equation \eqref{sd4}. In particular, we capture spillover effects by estimating a regression that includes a control for number of study respondents who listed this non-study respondent as a peer (i.e. the number of references) and the fraction of those references who are treatment respondents.  \footnote{%
Online Appendix A7 assesses whether the fraction of treated peers in a social
network is independent of other observable characteristics. The only
characteristic that shows an imbalance is cotton acreage. Controlling for baseline cotton acreage leaves the point estimates and their precision virtually unchanged.} }

\subsubsection{\protect\normalsize Among Study Respondents}

In column 4 of Table \ref{peer} we see that having a peer assigned to the treatment significantly increases the usage of AO on an extensive margin (13.4 p.p.) and on an intensive margin (91 additional minutes), suggesting that there are complementarities among respondents who have AO. 

However, we find little evidence to suggest that treated peers influence information sources used in agriculture, the adoption of inputs or sowing decisions, suggesting that these outcomes are unlikely to be underestimated by a large margin.


\subsubsection{\protect\normalsize Non-study Respondents}

{\normalsize Columns (5) and (6) of Table \ref{peer} refer to non-study respondents and report
simple differences using data collected from the peer survey. The
specification estimated here is Eq. \eqref{sd4},with controls for the number
of peers in one's reference group. This data provides a more powerful test for spillover effects, as non-study respondents are those who are listed by a study respondent at baseline but are not themselves a part of our study. As such, non-study respondents are, by definition, exposed to study respondents.}

{\normalsize Here, we do find some evidence for spillovers. Those with more treated peers in their networks also report 4\% less cotton crop loss as a result of pest attacks, suggesting that pest management practices provided by the AO service may have been shared. We do, however, acknowledge that this evidence is suggestive, insofar as we do not document systematic changes in pest management index among peer or, more generally, study respondent farmers. However, we find some encouragement in the fact that pest management was the most important topic covered in the service (see online Appendix Table A1).\footnote{Note, the peer survey did not collect information on seed choices or fertiliser use, hence we are not able to estimate these effects.}}

\section{\protect\normalsize Willingness to Pay and the Market for Information}

While the results above do not allow us to unambiguously establish the effect of AO on ultimate outcomes such as yield and profits, in this section, we assess farmer willingness to pay for the service. More generally, the financial sustainability of a subscription-based service would depend critically on users' willingness to pay. However, markets for information may face several important challenges. Potential buyers may not be able to evaluate the quality of the goods; information may be non-rival, allowing buyers to provide the information to neighbours (which may in turn suppress demand for the service). 

\subsection{Measuring Willingness to Pay} 

We measure willingness to pay (WTP) by offering for sale subscriptions at the conclusion of the study. We visited the 1,200 study respondents, as well as an additional 457 non-study respondents. One quarter of this sample (chosen at random, via scratchcards) was offered a fixed price for a nine-month subscription at a single price (“Take it or Leave it Offer”, or TIOLI); we randomly varied that price across farmers to estimate demand .\footnote{%
The prices offered were Rs. 40 (\$0.67), Rs. 90 (\$1.5), Rs. 140 (\$ 2.3),
Rs. 190 (\$3.2) and Rs. 240 (\$4).}\  For the remaining three-fourths of the sample, we used a modified version of the Becker-DeGroot-Marshack (BDM) method to elicit WTP. In this method, the respondent first indicates their willingness to purchase at a series of price points. The maximum WTP is recorded, after which a randomly generated offer price is drawn. If the offer price is below the bid, the sale takes place at the offer price; if the offer price is below the bid, the sale does not take place.\footnote{The respondent is asked to indicate their willingness to purchase the policy
for Rs. 40 (\$0.67), Rs. 90 (\$1.5), Rs. 140 (\$ 2.3), Rs. 190 (\$3.2), Rs.
240 (\$4), Rs. 290 (\$4.8), Rs. 390 (\$6.5), and Rs.490 (\$8.1)}

Overall, we find that 33\% of those offered a subscription purchased it across methods (see Panel A of online Appendix A16). Using the BDM method, we find that average WTP is Rs. 109 (\$ 1.79).\footnote{In this section and the following one, we use the conversion of 1 USD = INR 60.89} The two methods of eliciting WTP deliver similar results, with high-take up at low prices.\footnote{In \citet{cole2015elicitation} we document these findings across multiple experiments in detail.} Approximately 85\% of study respondents accepted the TIOLI price of 40, falling to 6.7\% at a price of Rs. 240. Acceptance rates for these prices were similar in the BDM sample (87\% at Rs. 40, 17\% at Rs. 240), and 6\% at Rs. 490. 

Restricting attention to study respondents, in Panel C of Table 5, we find the control group is willing to pay approximately Rs. 72 (\$1.18) on average, while those in the treatment group, who have experienced the service, are willing to pay approximately Rs. 18 more (significant at 5\% level). For the realised distribution of offer prices, AO users are 6 p.p. more likely to purchase a subscription. Adjusting for spillovers increases these treatment effects for both WTP (Rs. 21, significant at the 5\% level) and purchasing AO (6.6 p.p. significant at the 5\% level). Both of these effects suggest that exposure to the treatment resulted in an increased valuation of the service, two years after the fact.  

Online Appendix A16 examines several correlates of demand. In addition to treatment status, we find that those who used the service more and those with more education were more likely to purchase a subscription. 

\subsection{Profit Maximising Price} 

The distribution of willingness to pay from the BDM exercise in our study sample enables us to calculate the revenue-maximising price a private service provider might charge.  If there were approximately no cost to offering the service, a seller would maximise profits by charging Rs. 190 (\$3.12), and reach 25\% of the population. At a cost of Rs. 100 (what we estimate the roughly at-scale cost to be), a provider could maximise profits at Rs. 290 (\$4.76), and reach 25\% of the population. Finally, if the cost of providing the service were high (e.g., Rs. 200), profits would be maximised at a price of Rs. 490 (\$8.05).\footnote{In fact, a service provider might choose an even higher price, as we did not evaluate how demand drops when the price goes above Rs. 490 (\$ 8.05).}

We note that these are not true measures of the profitability of the service, as one would have to consider subscription renewal rates and customer acquisition costs, among other factors. Such an exercise is beyond the scope of this paper. 

\subsection{Cost-Benefit Analysis for Farmers} 

We estimate the costs of an AO subscription in this study at approximately Rs. 600 (\$10) per farmer per year, roughly comparable to a single in-person extension service,  Rs. 517 (\$8.50) per farmer per year.\footnote{Online Appendix A23 provides a breakdown of the costs of the service.} However, we note that the costs of a digital service would decline dramatically with scale with services similar to AO in India now pricing subscriptions at roughly \$2 per farmer per year. In addition, while the costs of airtime accounted for roughly 40\% of the cost of provision in our case, evolution in markets, such as data services in rural areas with little or no marginal cost, may considerably reduce the cost of offering services like this. 

As we discuss above, our results suggest that our study is likely not powered to detect meaningful gains in yields or profits; we are therefore not able to incorporate these estimates into a cost-benefit analysis. More generally, a key challenge in evaluating low-cost interventions are the large samples necessary to identify reasonable benefit cost ratios. One approach to this power problem is to combine results from multiple studies in a meta-analysis, as done by \citet{fabregas2019realizing} in a review article.  This article, surveying six studies in India and Africa that provide agricultural advice through digital technologies, they find an average impact on yields of approximately 4\%, with a 95\% confidence interval of (0,0.08).\footnote{Further, as \citet{fabregas2019realizing} note, the marginal cost of digital agricultural interventions they survey are so low that governments might seek to invest in these programs even where estimated positive returns are noisy unless they are very risk averse.}

In Table 5, we estimate an increase in input costs of roughly Rs. 1,800 (\$30). As such, a 4\% increase in yields would result in Rs. 4,689 (\$77) profit, net of the increase in input costs.\footnote{Here, we use total crop revenue for the control group at baseline: Rs.161,220 (\$2687)} For comparison, if yields were one standard deviation higher than the mean of the \citet{fabregas2019realizing} estimate (6 p.p.) the profit would be Rs. 7,976 (\$131), while if they were one standard deviation lower, profits would be Rs. 1,461 (\$24). In each case, this exceeds the Rs. 600 (\$10) cost of the subscription.\footnote{Were we to factor in the opportunity cost of a farmer's time spent on the AO service by using the median wage for an agricultural labourer, this would mean subtracting Rs. 76 (\$1.25) from the profit margin as the yearly use of the service amounts to roughly half a day's worth of labour.}

We note one final point: our evidence, and additional work by \citet{Fernando2016sotech}, suggests that farmers share a significant amount of information. While this may result in reduced estimates of WTP, this also suggests the importance of studying the externalities created by such a service, as we provide some evidence for above. 

\section{\protect\normalsize Threats to Validity}

\subsection{\protect\normalsize Attrition}

{\normalsize Online Appendix Table A8 analyzes the characteristics of attritors
from the study by treatment group. In the endline survey, we had 120
attritees, of which 39 were control farmers, 43 were from the AOE group, and
38 were from the AO group. In comparison, we had 77 attritees in the
midline, of which 23 were control farmers, 22 were from the AOE group, and
32 were from the AO group. For the most part, we do not observe any significant differences
between either the individual treatment arms or the combined treatment group (AOE+AO) and control group for the attritees, as measured by baseline characteristics. The one exception is in the case of wheat
cultivation between the AOE group and control at midline, in which AOE
attritees were less likely to grow wheat. This may in part explain why
respondents in the AOE group are 8\% more likely to cultivate wheat at
endline relative to the control group a result we do not emphasise because
it does not show a consistent pattern across treatment groups or across
rounds. }

\subsection{\protect\normalsize Experimenter Demand Effects}

{\normalsize A second concern is that respondents in the treatment group may
offer answers that they believe the research team seeks, perhaps in the
hopes of prolonging the research project or due to a sense of reciprocity.
While it is difficult to rule this out entirely, the fact that we find no
effect on sources of price information in Table \ref{info_gen} -- which the AO service does not provide -- in spite of
finding large differences for sources of other information supports the
interpretation that our results are not driven by demand effects.

We also note that we can observe some outcomes perfectly: the AO platform records
precisely how many times respondents call in. Respondents provide consistent answers to the question \textquotedblleft {}Did you call into the
AO line with a question,\textquotedblright {} with a 55.5\% self-reported
call-in rate vs. a 53.5\% call-in rate using administrative data (results
not reported in tables). Finally, we note that the survey teams did not
identify themselves as associated with the AO\ service.}

\subsection{\protect\normalsize Study Design}

An important limitation of the study is the within-village randomisation, which introduces the possibility of information spillovers. We address this concern in two ways. First, in Tables 3,4, and 5, we include terms that account for spillover effects as in equation \eqref{sd3}. As such, we can compare the estimated treated coefficient with (column 3) and without (column 2) spillover controls to understand the scope of the problem. For the most part, we do see some evidence that the treatment coefficients are lower when we don't account for spillover effects, though these differences are not typically economically large.  

Second, we estimate spillover effects directly by comparing control group respondents who are exposed to peers who are in the treatment group, to those who are not, as in Table 6. In online Appendix A24, using the same specification as in Table 6 (columns 2-4), we find that for all input indices, the coefficient on ‘Fraction of Peers Treated’ (column 3), which captures the effect of exposure to a treated peer for the control group, is rarely different from zero. We believe future research, using a clustered experimental design that includes `pure control' villages, could help more definitively address the importance of spillovers.

\section{\protect\normalsize Conclusion}

{\normalsize This paper presents the results from a randomised experiment
studying the impact of providing toll-free access to AO, a mobile
phone-based technology that allows farmers to receive timely agricultural
information from expert agronomists and their peers. }

{\normalsize Firstly, we show that the intervention was successful in
generating a substantial amount of AO usage, with roughly 60\% of all
treatment respondents calling in to listen to content or ask a question
within 7 months of beginning the intervention, and nearly 90\% using it
after two years. We then showed that AO had a large impact on reported
sources of information used in agricultural decisions. }

{\normalsize Having established AO as a reliable source of information, we
then show that advice provided through AO resulted in farmers consistently
changing their input decisions in cotton cultivation and, in particular, their seed choices in line with recommendations from the service. } 

{\normalsize We do not find evidence to suggest the service improved yields or profits. Unfortunately, the imprecision of our estimates suggests the study was underpowered to detect effects for these outcomes, suggesting a need for future research. In particular, the study of on-demand interventions will need to carefully consider the challenge in detecting treatment effects presented by a non-homogeneous treatment that is a function of important agricultural shocks like pest infestations and weather. }

{\normalsize These results suggest the importance of further research on the importance of information in smallholder
agriculture. Many important questions remain unanswered, such as: the
importance of the customized on-demand vs. \textquotedblleft
push\textquotedblright\ information; the role of existing trust in the NGO
associated with the service in promoting credibility; and the importance of
the \textquotedblleft unbiased\textquotedblright\ nature of the service
relative to commercially motivated information provided. We hope to test
these in future experiments.}

{\normalsize Finally, we stress the practical importance of this technology.
Climate change and the mono-cropping of new varieties of cotton may
significantly alter both the types and frequency of pests and the
effectiveness of pesticides in the near future. Farmers in isolated rural
areas have little recourse to scientific information that might allow them
to adapt to these contingencies. We believe mobile phone-based agricultural
extension presents a cost-effective and salient conduit through which to
relay such information.\vspace{50pt} }\

\noindent
Harvard Business School \\
University of Notre Dame

\newpage

\bibliographystyle{ejbib}
\bibliography{myrefs}

{\normalsize \newpage{} }

{\normalsize \setcounter{table}{0} \renewcommand{\thetable}{\arabic{table}} }

{\normalsize 
\begin{table}[tbp]
{\normalsize \centering
\begin{tabular}{|l|}
\hline
\textbf{baseline} \\ \hline
\end{tabular}
}
\caption{Table Caption}
\label{baseline}
\end{table}
}

{\normalsize 
\begin{table}[tbp]
{\normalsize \centering
\begin{tabular}{|l|}
\hline
\textbf{ao\_use} \\ \hline
\end{tabular}
}
\caption{Table Caption}
\label{ao_use}
\end{table}
}

{\normalsize 
\begin{table}[tbp]
{\normalsize \centering
\begin{tabular}{|l|}
\hline
\textbf{info\_gen} \\ \hline
\end{tabular}
}
\caption{Table Caption}
\label{info_gen}
\end{table}
}

{\normalsize 
\begin{table}[tbp]
{\normalsize \centering
\begin{tabular}{|l|}
\hline
\textbf{input} \\ \hline
\end{tabular}
}
\caption{Table Caption}
\label{input}
\end{table}
}

{\normalsize 
\begin{table}[tbp]
{\normalsize \centering
\begin{tabular}{|l|}
\hline
\textbf{sowing} \\ \hline
\end{tabular}
}
\caption{Table Caption}
\label{sowing}
\end{table}
}


{\normalsize 
\begin{table}[tbp]
{\normalsize \centering
\begin{tabular}{|l|}
\hline
\textbf{peer} \\ \hline
\end{tabular}
}
\caption{Table Caption}
\label{peer}
\end{table}
}



\end{document}
